% !TEX TS-program = pdflatex
% !TEX encoding = UTF-8 Unicode

% This is a simple template for a LaTeX document using the "article" class.
% See "book", "report", "letter" for other types of document.

\documentclass[12pt]{article} % use larger type; default would be 10pt

\usepackage[utf8]{inputenc} % set input encoding (not needed with XeLaTeX)

%%% Examples of Article customizations
% These packages are optional, depending whether you want the features they provide.
% See the LaTeX Companion or other references for full information.

%%% PAGE DIMENSIONS
\usepackage{geometry} % to change the page dimensions
\geometry{a4paper} % or letterpaper (US) or a5paper or....
\usepackage{setspace}
\usepackage{parskip}
\parskip=1\baselineskip %\advance\parskip by 0pt plus 2pt% to change between paragraphs space
% \geometry{margin=2in} % for example, change the margins to 2 inches all round
% \geometry{landscape} % set up the page for landscape
%   read geometry.pdf for detailed page layout information

% \usepackage{gravarphicx} % support the \includegravarphics command and options

% \usepackage[parfill]{parskip} % Activate to begin paragraphs with an empty line rather than an indent

%%% PACKAGES
\usepackage{booktabs} % for much better looking tables
\usepackage{array} % for better arrays (eg matrices) in maths
\usepackage{paralist} % very flexible & customisable lists (eg. enumerate/itemize, etc.)
\usepackage{verbatim} % adds environment for commenting out blocks of text & for better verbatim
\usepackage{subfig} % make it possible to include more than one captioned figure/table in a single float
% These packages are all incorporated in the memoir class to one degree or another...
\usepackage[fleqn]{amsmath}
\usepackage{amssymb}
\usepackage{enumitem}
\usepackage{amsthm}
\usepackage{graphicx}
\usepackage{filecontents}
\usepackage{natbib}
\usepackage{scrextend}
\usepackage[table,xcdraw]{xcolor}
\usepackage{tabularx}
% If you use beamer only pass "xcolor=table" option, i.e. \documentclass[xcolor=table]{beamer}
\usepackage{lscape}
\usepackage{longtable}
 \usepackage{booktabs}




%%% HEADERS & FOOTERS
\usepackage{fancyhdr} % This should be set AFTER setting up the page geometry
\pagestyle{fancy} % options: empty , plain , fancy
\renewcommand{\headrulewidth}{0pt} % customise the layout...
\lhead{}\chead{}\rhead{}
\lfoot{}\cfoot{\thepage}\rfoot{}

%%% SECTION TITLE APPEARANCE
\usepackage{sectsty}
\allsectionsfont{\rmfamily\bfseries\upshape} % (See the fntguide.pdf for font help)
% (This matches ConTeXt defaults)

%%% ToC (table of contents) APPEARANCE
\usepackage[nottoc,notlof,notlot]{tocbibind} % Put the bibliography in the ToC
\usepackage[titles,subfigure]{tocloft} % Alter the style of the Table of Contents
\renewcommand{\cftsecfont}{\rmfamily\mdseries\upshape}
\renewcommand{\cftsecpagefont}{\rmfamily\mdseries\upshape} % No bold!

\usepackage[colorlinks,citecolor=black,urlcolor=black,bookmarks=false,hypertexnames=true]{hyperref} 
%%% END Article customizations
%%% The "real" document content comes below...

\title{Predicting Housing Prices in Denmark\\ A Machine Learning Approach}
\author{Group 23: He SHI, Mingjie XING, Qi ZHANG\\Department of Economics, University of Copenhagen Summer 2023}
\date{\today} % Activate to display a given date or no date (if empty),
         % otherwise the current date is printed 

\begin{document}
\maketitle


\begin{enumerate}

  \item What is your research question?\par
  		This project is intended to build a predictor of residential housing prices in cities in Denmark with a machine learning hedonic model based on interior features such as house area and location, and exterior features such as income level of the neighbourhood, weather condition and cities' economic development level. The project mainly covers major cities in Denmark, e.g., Copenhagen and Aarhus. The frequency of the data will be monthly. The trained model based on 2023 data will be tested on 2022 and 2021 data. 

  \item What kind of data are you planning on using? How will you get access to these data?\par
  	%\begin{itemize}
	%	\item House Prices Data: \par
			We are going to scrape the individual level house prices from boligsiden, a major real estate broker in Denmark covering housing prices as monthly data of 36 months scale and detailed features of houses in different regions with an open data access. We will scrape the data from \url{https://www.boligsiden.dk/}. Structural features in data includes area, location, age and owner income. Web-scraping of the data is allowed for academic use. There will be about 10000 house-month-region observations in our dataset. 
		%\item Income Data:\par
		%	Pay level distribution by region can be found on STATISTICS DENMARK under the Pay Level subsection of Income and Earnings. We are not going to carry out time series analysis due to the constraint of available price data.
	%\end{itemize}

  \item What will your data analysis be like? Will you use machine learning? How?\par
		We are going to train a machine learning model on residential housing prices in Copenhagen in 2023 in each postal coded region. The feature variables include structural features such as house area and age, and external variables include location, average income of the neighbourhood and weather conditions of each city. We are going to test the model with 2022 data and data from other cities such as Aarhus.\par
		We are going to apply the polynomial and LASSO regression on the data to obtain the coefficient of deciding factors on house price in different cities in Denmark and in different temporal sphere.\par
		We will handle the under-fitting and over-fitting problem when applying the machine learning model to get a more precise version of the house price prediction.\par
		We will conduct cross-validation with k-fold method.\par
		
  \item Have you already identified other papers within this area that you can use in a literature review? If so, name a few and explain what they do in one sentence only.\par
		There are several papers recently on top journals in real estate economics predicting housing prices with machine learning models: \newline
		\cite{Deppneretal2023} applies non-parametric machine learning model with k-fold cross-validation to examine the U.S. commercial real estate appraisal to shrink the deviations between market values and subsequent transaction prices. \newline
		\cite{Linetal2023} augments the traditional hedonic model with Gradient Boosting Machine algorithm to predict housing prices in Beijing. \newline
		\cite{TchuenteNyawa2022} uses artificial neural networks, random forest, adaptive bBoosting, gradient boosting and K-nearest neighbours algorithms to estimate real estate prices in French cities.
		
  \item How do you 'contribute' to the literature?\par
		The main contribution of this project includes: 1) creating a comprehensive dataset on Denmark residential real estate market; 2) building a predictor of Denmark housing prices in different cities.
\end{enumerate}
	
\newpage	
\footnotesize
\bibliographystyle{apalike}
\bibliography{ED}


\end{document}
