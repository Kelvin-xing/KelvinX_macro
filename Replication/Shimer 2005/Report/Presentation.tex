\documentclass{beamer}
\usepackage{xmpmulti}
\usepackage{amsmath}
\usepackage{booktabs}
\usepackage{csquotes}% Recommended
\usepackage{natbib}
\usepackage{hyperref}
\usepackage{bbm}
\usepackage{comment}
\usepackage{bm}
\usepackage{makecell}
\usepackage{url}
\setbeamertemplate{footline}[frame number]



\AtBeginSection[]{
    \begin{frame}
        \tableofcontents[currentsection]
    \end{frame}
}

\begin{comment}
    \AtBeginSubsection[]{
        \begin{frame}
            \tableofcontents[currentsection]
        \end{frame}
    }
\end{comment}


\title{The Cyclical Behavior of Equilibrium Unemployment and Vacancies\\
\citet{Shimer2005}}
\author{Presented by: XING Mingjie}
\date{\today}
\begin{document}

\begin{frame}
  \titlepage
\end{frame}

\begin{frame}{Bellman Equation}{Worker}
    \begin{itemize}
        \item Unempolyment value
            \begin{equation}\label{BellmanUnemploy}
                U_p = z + \delta \{f(\theta_p)\mathbb{E}_p W_{p^\prime} + (1-f(\theta_p))\mathbb{E}_p U_{p^\prime}\}
            \end{equation}
        \item Employment value
            \begin{equation}\label{BellmanEmploy}
                W_p = w_p + \delta \{(1-s)\mathbb{E}_p W_{p^\prime} + s\mathbb{E}_p U_{p^\prime}\}
            \end{equation}
    \end{itemize}
\end{frame}

\begin{frame}{Bellman Equation}{Firm}
    \begin{itemize}
        \item Hiring value
            \begin{equation}\label{BellmanHire}
                J_p = p - w_p + \delta (1-s)\mathbb{E}_p J_{p^\prime}
            \end{equation}
        \item Vacancy value
            \begin{equation}\label{BellmanVacancy}
                V_p = -c + \delta q(\theta_p)\mathbb{E}_p J_{p^\prime} \equiv 0
            \end{equation}
    \end{itemize}
\end{frame}

\begin{frame}{Productivity}
    The log of productivity follows AR(1) process
    \begin{equation}\label{productivityprocess}
    \log(p) = \rho \log(p) + \varepsilon 
    \end{equation}
    where \[\log(p) \sim N(\mu_\lambda, \sigma^2_\lambda),\ 
    \varepsilon \sim N(\mu_\varepsilon,\sigma^2_\varepsilon)\]
\end{frame}


\begin{frame}{Optimal Control}{Market tightness}
    \begin{itemize}
        \item Control in this problem consists of \(w_p, \theta_p, u_p\) and the state is \(p\)
        \item Market tightness \(\theta_p\) is given by solving the following equation of hire rate from free entry condition 
            \begin{equation}\label{HireRate}
            q(\theta_p) = \frac{c}{\delta \mathbb{E}_p J_{p^\prime}}\end{equation}
        \item And market tightness \begin{equation}
            \label{MarketTightness}
            \theta_p = (\frac{q(\theta_p)}{\mu})^{-\frac{1}{\eta}}
            \end{equation}
        \item Employ Rate is given by \begin{equation}\label{EmployRate}
            f(\theta_p) = \mu^\frac{1}{\eta} q^\frac{\eta-1}{\eta}
        \end{equation}
    \end{itemize}
\end{frame}

\begin{frame}{Optimal Control}{Continued}
    \begin{itemize}
        \item Optimal wage at each productivity level is given by the Nash Bargaining: 
        \begin{equation}\label{Nash}W_p - U_p = \beta (W_p - U_p + J_p)\end{equation}
        \item Note Bellman Equation of \(W_p\) given by \ref{BellmanEmploy}, \(U_p\) given by \ref{BellmanUnemploy}, \(J_p\) given by \ref{BellmanHire}
        \item Following the algebra given in slide \ref{OptWage}, optimal wage for each \(p\) is \begin{equation}\label{OptimalWage}
            w_p = \beta p + (1-\beta)z + \beta c \theta_p\end{equation}
        \item And unemployment rate \begin{equation}\label{UnemployRate}
            u_p = \frac{\delta}{\delta + f(\theta_p)}
        \end{equation}
    \end{itemize}
\end{frame}

\begin{frame}[allowframebreaks]{Question a}{Descretization Algorithm}
    Inspired by Karen A. Kopecky 2006 Lecture Note
    \begin{enumerate}
        \item Choose a relateive error tolerance level \texttt{tol};
        \item Discretize the state space by constructing a grid for productivity \[
            p = \exp\{logp\}
            \text{\ where\ } logp = \{logp_1, logp_2, \ldots, logp_n\} \] given by the Tauchen-Hussey (1991) method.
            The n is chosen at 100;
        \item Start with an initial guess of the value function \(V^{(0)}(p)\) is a vector of length \(n\), i.e., \(V^{(0)}(p) = \{V^{(0)}_i\}_{i=1}^n\), where \(V^{(0)}_i = V^{(0)}(p_i)\).
            \(V\) here represents \(U, W, J\). The initial guess is ones.
        \framebreak
        \item Update the value function using eqautions \ref{BellmanUnemploy} to \ref{OptimalWage}, specifically
            \begin{enumerate}
                \item Fix the current productivity level at one of the grid points, \(p_i\) from \(i=1\)
                \item For each possible choice of productivity next period, calculate optimal control in the following order:
                \begin{gather*}
                    q(\theta_{p_i}) = \frac{c}{\delta \sum_{j=1}^{n}p_{i,j}J^{(0)}(p_j)}\\
                    f(\theta_{p_i}) = \mu^\frac{1}{\eta} q^\frac{\eta-1}{\eta}\\
                    \theta_{p_i} = (\frac{q(\theta_{p_i})}{\mu})^{-\frac{1}{\eta}}\\
                    w_{p_i} = \beta p_i + (1-\beta)z + \beta c \theta_{p_i}
                \end{gather*}
                \item and update the value function system with 
                \framebreak
                \begin{gather*}
                    U^{(1)}_{p_i} = z + \delta\{f(\theta_{p_i})\sum_{j=1}^{n}p_{i,j}W^{(0)}(p_j) + (1-f(\theta_{p_i}))\sum_{j=1}^{n}p_{i,j}U^{(0)}(p_j)\}\\
                    W^{(1)}_{p_i} = w_{p_i} + \delta\{(1-s)\sum_{j=1}^{n}p_{i,j}W^{(0)}(p_j) + s\sum_{j=1}^{n}p_{i,j}U^{(0)}(p_j)\}\\
                    J^{(1)}_{p_i} = p_i - w_{p_i} + \delta (1-s)\sum_{j=1}^{n}p_{i,j}J^{(0)}(p_j)
                \end{gather*}
                \item Choose a new grid point for productivity, go through 4.1 to 4.3. Once we have done the update for all productivity grid, we have new system of value function \({V^{(1)}_p}\)
                \item Compute distance between the two systems of value functions following the sup norm \[
                    d = \max\limits_{i\in\{1,\ldots,n\}}|V^{(0)}_i-V^{(1)}_i|\]
                \framebreak
                \item If distance is within the error tolerance level, \(d \leq tol * ||V^{(1)}_1||\), the functions have converged and go to step 5, or else go back to step 4.
            \end{enumerate}
            \item Calculate the optimal control for each productivity level:
            \begin{gather*}
                q(\theta^*_{p_i}) = \frac{c}{\delta \sum_{j=1}^{n}p_{i,j}J^* (p_j)}\\
                f(\theta^*_{p_i}) = \mu^\frac{1}{\eta} q^\frac{\eta-1}{\eta}\\
                \theta^*_{p_i} = (\frac{q(\theta^*_{p_i})}{\mu})^{-\frac{1}{\eta}}\\
                w^*_{p_i} = \beta p_i + (1-\beta)z + \beta c \theta^*_{p_i}\\
                u^*_p = \frac{\delta}{\delta + f(\theta^*_p)}
            \end{gather*}
            where \(J^*\) is the converged value function.
    \end{enumerate}
\end{frame}

\begin{frame}
    The Hermite roots and weights are produced with \cite{VanDamme2023} and checked with \cite{Salzer1952}. The results comes with 4 decimals.
\end{frame}

\begin{frame}{Calibration}
    
\end{frame}
    

\begin{frame}{Appendix A}{Optimal wage}\label{OptWage}
    \begin{gather*}
        W_p - U_p = \beta (W_p - U_p + J_p)\\
        \Leftrightarrow
        w_p - z + \delta (1-s-f(\theta_p))(\mathbb{E}_p W_{p^\prime}-\mathbb{E}_p U_{p^\prime}) =\\
        \beta (p-z+\delta (1-s-f(\theta_p))(\mathbb{E}_p W_{p^\prime}-\mathbb{E}_p U_{p^\prime}) +\delta(1-s)\mathbb{E}_p J_{p^\prime})\\
        \Leftrightarrow
        w_p = \beta p + (1-\beta)z + (\beta-1)\delta(1-s-f(\theta_p))(\mathbb{E}_p W_{p^\prime}-\mathbb{E}_p U_{p^\prime})\\
         + \frac{\beta c(1-s)}{q(\theta_p)}\\
        \Leftrightarrow
        w_p = \beta p + (1-\beta)z - \frac{\beta c\delta(1-s-f(\theta_p))}{q(\theta_p)} + \frac{\beta c(1-s)}{q(\theta_p)}\\
        \Leftrightarrow
        w_p = \beta p + (1-\beta)z + \beta c \theta_p
    \end{gather*}
    where we use the fact that \(\mathbb{E}_p W_{p^\prime}-\mathbb{E}_p U_{p^\prime} = \frac{\beta}{1-\beta}\mathbb{E}_p J_{p^\prime}\) and \(f(\theta_p)/q(\theta_p)=\theta_p\)
\end{frame}

\setbeamertemplate{footline}{}
\begin{frame}[allowframebreaks,noframenumbering]{Reference}
    \footnotesize
    \bibliographystyle{agsm}
\bibliography{Shimer2005}
\end{frame}
\end{document}