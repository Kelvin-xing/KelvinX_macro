\documentclass{beamer}
\usepackage{xmpmulti}
\usepackage{amsmath}
\usepackage{booktabs}
\usepackage{csquotes}% Recommended
\usepackage{natbib}
\usepackage{hyperref}
\usepackage{bbm}
\usepackage{comment}
\usepackage{bm}
\usepackage{makecell}
\usepackage{url}
\setbeamertemplate{footline}[frame number]



\AtBeginSection[]{
    \begin{frame}
        \tableofcontents[currentsection]
    \end{frame}
}

\begin{comment}
    \AtBeginSubsection[]{
        \begin{frame}
            \tableofcontents[currentsection]
        \end{frame}
    }
\end{comment}


\title{The Cyclical Behavior of Equilibrium Unemployment and Vacancies\\
\citet{Shimer2005}}
\author{Presented by: XING Mingjie}
\date{\today}
\begin{document}

\begin{frame}
  \titlepage
\end{frame}

\begin{frame}{Bellman Equation}{Worker}
    \begin{itemize}
        \item Unempolyment value
            \begin{equation}\label{BellmanUnemploy}
                U_p = z + \delta \{f(\theta_p)\mathbb{E}_p W_{p^\prime} + (1-f(\theta_p))\mathbb{E}_p U_{p^\prime}\}
            \end{equation}
        \item Employment value
            \begin{equation}\label{BellmanEmploy}
                W_p = w_p + \delta \{(1-s))\mathbb{E}_p W_{p^\prime} + s\mathbb{E}_p U_{p^\prime}\}
            \end{equation}
    \end{itemize}
\end{frame}

\begin{frame}{Bellman Equation}{Firm}
    \begin{itemize}
        \item Hiring value
            \begin{equation}\label{BellmanHire}
                J_p = p - w_p + \delta (1-s)\mathbb{E}_p J_{p^\prime}
            \end{equation}
        \item Vacancy value
            \begin{equation}\label{BellmanVacancy}
                V_p = -c + \delta q(\theta_p)\mathbb{E}_p J_{p^\prime} \equiv 0
            \end{equation}
    \end{itemize}
\end{frame}

\begin{frame}{Optimal Control}{Market tightness}
    \begin{itemize}
        \item Control in this problem consists of \(w_p, \theta_p, u_p\) and the state is \(p\)
        \item Market tightness \(\theta_p\) is given by solving following equation for hire rate from free entry condition \begin{equation}\label{HireRate}
            q(\theta_p) = \frac{c}{\delta \mathbb{E}_p J_{p^\prime}}\end{equation}
        \item And market tightness \begin{equation}
            \label{MarketTightness}
            \theta_p = (\frac{q(\theta_p)}{\mu})^{-\frac{1}{\eta}}
            \end{equation}
        \item Employ Rate is given by \begin{equation}\label{EmployRate}
            f(\theta_p) = \mu^\frac{1}{\eta} q^\frac{\eta-1}{\eta}
        \end{equation}
    \end{itemize}
\end{frame}

\begin{frame}{Optimal Control}{Continued}
    \begin{itemize}
        \item Optimal wage at each productivity level is given by the Nash Bargaining: 
        \begin{equation}W_p - U_p = \beta (W_p - U_p + J_p)\end{equation}
        \item Note Bellman Equation of \(W_p\) given by \ref{BellmanEmploy}, \(U_p\) given by \ref{BellmanUnemploy}, \(J_p\) given by \ref{BellmanHire}
        \item Following the algebra given in slide \ref{OptWage}, optimal wage for each \(p\) is \[
            w_p = \beta p + (1-\beta)z + \beta c \theta_p\]
        \item And unemployment rate \begin{equation}\label{UnemployRate}
            u_p = \frac{\delta}{\delta + f(\theta_p)}
        \end{equation}
    \end{itemize}
\end{frame}

\begin{frame}{Appendix A}{Optimal wage}\label{OptWage}

    a
\end{frame}

\setbeamertemplate{footline}{}
\begin{frame}[allowframebreaks,noframenumbering]{Reference}
    \footnotesize
    \bibliographystyle{agsm}
\bibliography{Shimer2005}
\end{frame}
\end{document}