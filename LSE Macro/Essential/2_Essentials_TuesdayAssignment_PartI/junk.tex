%2multibyte Version: 5.50.0.2953 CodePage: 1252

\documentclass{article}
%%%%%%%%%%%%%%%%%%%%%%%%%%%%%%%%%%%%%%%%%%%%%%%%%%%%%%%%%%%%%%%%%%%%%%%%%%%%%%%%%%%%%%%%%%%%%%%%%%%%%%%%%%%%%%%%%%%%%%%%%%%%%%%%%%%%%%%%%%%%%%%%%%%%%%%%%%%%%%%%%%%%%%%%%%%%%%%%%%%%%%%%%%%%%%%%%%%%%%%%%%%%%%%%%%%%%%%%%%%%%%%%%%%%%%%%%%%%%%%%%%%%%%%%%%%%
\usepackage{amsmath}
\usepackage{setspace}

\setcounter{MaxMatrixCols}{10}
%TCIDATA{OutputFilter=LATEX.DLL}
%TCIDATA{Version=5.50.0.2953}
%TCIDATA{Codepage=1252}
%TCIDATA{<META NAME="SaveForMode" CONTENT="1">}
%TCIDATA{BibliographyScheme=Manual}
%TCIDATA{Created=Tuesday, July 27, 2010 12:19:57}
%TCIDATA{LastRevised=Saturday, August 07, 2010 13:11:44}
%TCIDATA{<META NAME="GraphicsSave" CONTENT="32">}
%TCIDATA{<META NAME="DocumentShell" CONTENT="Standard LaTeX\Blank - Standard LaTeX Article">}
%TCIDATA{Language=American English}
%TCIDATA{CSTFile=40 LaTeX article.cst}

\setlength{\textwidth}{160mm}
\setlength{\textheight}{210mm}
\setlength{\oddsidemargin}{5mm}
\setlength{\evensidemargin}{5mm}
\setlength{\topmargin}{-5mm}
\newtheorem{theorem}{Theorem}
\newtheorem{acknowledgement}[theorem]{Acknowledgement}
\newtheorem{algorithm}[theorem]{Algorithm}
\newtheorem{axiom}[theorem]{Axiom}
\newtheorem{case}[theorem]{Case}
\newtheorem{claim}[theorem]{Claim}
\newtheorem{conclusion}[theorem]{Conclusion}
\newtheorem{condition}[theorem]{Condition}
\newtheorem{conjecture}[theorem]{Conjecture}
\newtheorem{corollary}[theorem]{Corollary}
\newtheorem{criterion}[theorem]{Criterion}
\newtheorem{definition}[theorem]{Definition}
\newtheorem{example}[theorem]{Example}
\newtheorem{exercise}[theorem]{Exercise}
\newtheorem{lemma}[theorem]{Lemma}
\newtheorem{notation}[theorem]{Notation}
\newtheorem{problem}[theorem]{Problem}
\newtheorem{proposition}[theorem]{Proposition}
\newtheorem{remark}[theorem]{Remark}
\newtheorem{solution}[theorem]{Solution}
\newtheorem{summary}[theorem]{Summary}
\newenvironment{proof}[1][Proof]{\noindent\textbf{#1.} }{\ \rule{0.5em}{0.5em}}
%\input{tcilatex}
\begin{document}


\begin{center}
\textbf{Amsterdam Macroeconomics Summer School 2010}

\textbf{Part I: The Essentials}

\textbf{University of Amsterdam\bigskip }

\textbf{Tuesday Assignment}

\textbf{Solving a model with time-varying risk premium with projection
methods\bigskip }
\end{center}

\section{Goal}

The goal of this assignment is to show that solving a model with projection
methods can actually be quite simple. There is only one somewhat tricky
step, namely evaluating the conditional expectation using numerical
integration.

\section{Model}

The solution to the Lucas Asset Pricing Model is characterized as follows%
\begin{eqnarray*}
p_{t} &=&\beta \text{E}_{t}\left( \left( \frac{c_{t+1}}{c_{t}}\right)
^{-\gamma }\left( d_{t+1}+p_{t+1}\right) \right) \\
c_{t} &=&d_{t} \\
d_{t} &=&\mu _{d}+\rho _{d}d_{t-1}+\varepsilon _{t} \\
\varepsilon _{t} &\sim &N\left( 0,\sigma ^{2}\right)
\end{eqnarray*}%
Of course, you can choose your own parameters, but to make our programs
comparable consider\newline
%TCIMACRO{\TeXButton{center}{\begin{center}}}%
%BeginExpansion
\begin{center}%
%EndExpansion
\begin{tabular}{l|lllll}
Parameter & $\mu _{d}$ & $\rho _{d}$ & $\sigma $ & $\beta $ & $\gamma $ \\ 
\hline
Value & $0.1$ & $0.9$ & $0.1$ & $0.9$ & $3$.%
\end{tabular}%
%TCIMACRO{\TeXButton{center}{\end{center}}}%
%BeginExpansion
\end{center}%
%EndExpansion

\section{Preliminaries}

\subsection{Approximating function}

You always have to take a stand on which function you want to approximate
and with what kind of approximating function. In this assignment, you should
approximate the price function using the nth-order polynomial%
\begin{equation*}
p_{t}\left( d_{t}\right) =\sum_{i=0}^{n}a_{i}d_{t}^{i}\text{.}
\end{equation*}%
The Matlab function pfunc.m, included with the assignment, defines this
function.

\subsection{Grid}

With projection methods, you have to create a grid for the state variables.
In the Lucas Asset Pricing Model, $d_{t}$ is the only state variable and we
can calculate the mean and standard deviation. In this assignment, we
construct a grid for $d_{t}$ ranging from three standard deviations below
the mean to three standard deviations above the mean.

\subsection{Numerical integration}

The tricky step is to evaluate the conditional expectation using numerical
integration. But don't be scared of this. Just realize the following:

\begin{itemize}
\item The stochastic variable $\varepsilon _{t+1}$ shows up three times
inside the conditional expecation. Once because $d_{t+1}$ depends on $%
\varepsilon _{t+1}$, once because $c_{t+1}$ depends on $d_{t+1}$ which in
turn depends on $\varepsilon _{t+1}$, and once because $p_{t+1}$ depends on $%
d_{t+1}$ which in turn depends on $\varepsilon _{t+1}$.

\item Numerical integration basically comes down to thinking of $\varepsilon
_{t+1}$ as a random variable with discrete support with each realization
having a probability. Remember from the slides that if $x\sim N\left( \mu
,\sigma ^{2}\right) $,%
\begin{equation*}
\text{E}\left( h\left( x\right) \right) \approx \sum_{i=1}^{gh}\left( \frac{%
\omega _{i}}{\sqrt{\pi }}h\left( \mu +\sigma \sqrt{2}\zeta _{i}\right)
\right)
\end{equation*}%
where $\omega _{i}$ and $\zeta _{i}$ are the Gaussian Hermite weights, sort
of like probabilities and nodes respectively.
\end{itemize}

\section{Exercise 1: solve for p}

\begin{enumerate}
\item Complete the external function that calculates the sum of squared
Euler equation errors. This is found in the included Matlab file \texttt{%
errfunc.m}.

\item The main projection program is given in the file main.m. Understand
the structure of this file. Note that it really does not do much more than
construct a grid and call a minimization routine. Run this projection
algorithm. Now you have solved your first projection problem!

\item Plot the policy function. Hint: you can use the pfunc.m function.%
\newline
\emph{!!! The coefficients used as input in pfunc.m must be ordered in the
right way. The first element is the element corresponding to the highest
order and the last element is the constant.}
\end{enumerate}

\section{Expected risk premium}

The expected risk premium $x_{t}$ is the difference between the expected
rate of return E$_{t}\left( r_{t+1}\right) $ and the risk-free rate$%
~r_{t}^{f}$. The two rates and the premium are given by the following
equations%
\begin{eqnarray*}
1 &=&\left( 1+r_{t}^{f}\right) \beta \text{E}_{t}\left( \left( \frac{c_{t+1}%
}{c_{t}}\right) ^{-\gamma }\right) \\
r_{t+1} &=&\frac{d_{t+1}+p_{t+1}}{p_{t}}-1 \\
x_{t} &=&\text{E}_{t}\left( r_{t+1}\right) -r_{t}^{f}\text{.}
\end{eqnarray*}%
We could have solved for $x_{t}$ simultaneously with $p_{t}$. But given the
solution for $p_{t}$, it is very easy to solve for $x_{t}$. You are asked to
approximate the expected risk premium function using the nth-order polynomial%
\begin{equation*}
x_{t}\left( d_{t}\right) =\sum_{i=0}^{n}b_{i}d_{t}^{i}\text{.}
\end{equation*}%
The Matlab function \texttt{xfunc.m} defines this function.

\section{Exercise 2: solve for x}

\begin{enumerate}
\item To solve for the values of $b_{i}$, i.e. the coefficients of the
approximating function for the expected risk premium, you do \emph{not} have
to use a minimization routine. You can simply calculate%
\begin{equation*}
x_{t}=\text{E}_{t}\left( r_{t+1}\right) -r_{t}^{f}
\end{equation*}%
at each grid point, where $r_{t}^{f}$ is defined by%
\begin{equation*}
r_{t}^{f}=\frac{1}{\beta \text{E}_{t}\left( \left( \frac{c_{t+1}}{c_{t}}%
\right) ^{-\gamma }\right) }-1
\end{equation*}%
and then find the coefficients by running a projection (regression).

\item Use your solution for $x_{t}(d_{t})$ to evaluate how much
time-variation there is in the expected risk premium.

\item How would the simulation for the expected risk premium look like if
you would use first-order perturbation? And if you would use second-order
perturbation?
\end{enumerate}

\end{document}
