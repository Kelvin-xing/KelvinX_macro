%Needed to produce a figure
%\usepackage{savetrees}
%\usepackage[hmargin=2cm,vmargin=2.5cm]{geometry}
%\pdfpagewidth 8.5in
%\pdfpageheight 11in 


\documentclass{article}
%%%%%%%%%%%%%%%%%%%%%%%%%%%%%%%%%%%%%%%%%%%%%%%%%%%%%%%%%%%%%%%%%%%%%%%%%%%%%%%%%%%%%%%%%%%%%%%%%%%%%%%%%%%%%%%%%%%%%%%%%%%%%%%%%%%%%%%%%%%%%%%%%%%%%%%%%%%%%%%%%%%%%%%%%%%%%%%%%%%%%%%%%%%%%%%%%%%%%%%%%%%%%%%%%%%%%%%%%%%%%%%%%%%%%%%%%%%%%%%%%%%%%%%%%%%%
\usepackage{graphicx}
\usepackage{multirow}
\usepackage{footmisc}
\usepackage{amsmath}
\usepackage{subfigure}
\usepackage{rotating}
\usepackage{epsfig}
\usepackage{float}

\setcounter{MaxMatrixCols}{10}
%TCIDATA{OutputFilter=Latex.dll}
%TCIDATA{Version=5.50.0.2953}
%TCIDATA{<META NAME="SaveForMode" CONTENT="1">}
%TCIDATA{BibliographyScheme=Manual}
%TCIDATA{LastRevised=Thursday, July 29, 2010 11:09:10}
%TCIDATA{<META NAME="GraphicsSave" CONTENT="32">}

\widowpenalty=3000000
\clubpenalty=3000000

%\input{tcilatex}

\begin{document}


\begin{center}
\textbf{Amsterdam Macroeconomics Summer School 2010}

\textbf{Part I: The Essentials}

\textbf{University of Amsterdam\bigskip }

\textbf{Wednesday Assignment}

\textbf{1. The difference between linear and loglinear perturbation}

\textbf{2. The difference between first-order and seconder-order perturbation%
}

\textbf{3. Calculate IRFs for non-linear models}

\textbf{4. TFP versus investment-specific technology shocks}{\huge \\[0pt]
}
\end{center}

\section{Objective}

The objectives of this assignment are the following:

\begin{enumerate}
\item Understand that there are several first-order perturbation solutions.
Here we focus on two of them, namely linear in levels and linear in logs.

\item We will investigate whether a second-order perturbation solution is
different from a first-order perturbation solution.

\item Learn how to calculate IRFs for non-linear policy functions.

\item We will investigate the behavior of a standard RBC model when there
are TFP shocks and when there are investment specific shocks. Several
articles point at the importance of investment-specific shocks (e.g. Jonas
Fisher's 2006 JPE paper). This paper shows that in a \emph{standard} RBC
model investment-specific shocks have some odd implications.
\end{enumerate}

\section{Comparing perturbation solutions}

\noindent Consider the following RBC model:

\begin{gather*}
\max_{\left\{ c_{t},k_{t}\right\} _{t=1}^{\infty }}\hspace{2mm}\displaystyle%
\text{E}\sum_{t=1}^{\infty }\beta ^{t}\frac{c_{t}^{1-\nu }}{1-\nu }, \\
\text{s.t.}
\end{gather*}

\begin{equation}
c_{t}+k_{t}=z_{t}k_{t-1}^{\alpha }+(1-\delta )k_{t-1}
\end{equation}%
\begin{equation}
z_{t}=(1-\rho )+\rho z_{t-1}+e_{t}
\end{equation}

\noindent The first-order condition for this problem is: 
\begin{equation}
c_{t}^{-\nu }=\text{E}_{t}\left[ \beta c_{t+1}^{-\nu }\left( z_{t+1}\alpha
k_{t}^{\alpha -1}+1-\delta \right) \right] 
\end{equation}

You will solve this model using perturbation methods in four ways:

\begin{itemize}
\item 1st order approximation in levels

\item 2nd order approximation in levels

\item 1st order approximation in logs

\item 2nd order approximation in logs
\end{itemize}

The Matlab file \texttt{Perturbation.m} is the master file. In particular,
it sets the values of $\nu $ and $\sigma $ (and the Dynare files will use
these). These two parameters are the key parameters for this problem because
the problem becomes more nonlinear if the values of these parameters
increase. The files \texttt{RBClevels1st.mod}  and \texttt{RBClevels2nd.mod}
are the Dynare programs to generate a polynomial solution in levels for
first and second-order, respectively. 

You are asked to do the following:

\begin{enumerate}
\item Using the provided *.mod files as an example write the Dynare files
that will generate a solution in \emph{log levels}. !!! Do not change the
law of motion for $z_{t}$. That is, you should only consider transformations
of $c_{t}$ and $k_{t}$.

\item Open the main file, \texttt{Perturbation.m}. It already generates
artificial data series for \texttt{RBClevels1st.mod} and \texttt{%
RBClogs1st.mod} (i.e., the file you have just created). What you are asked
to do is to write the part to generate  artificial data from decision rules
obtained from the two second-order approximations.

\item In the program provided the values of $\sigma $ and $\nu $ are set
equal to 0.007 and $3$, two very reasonable values. Investigate what happens
if you consider different values.\newline
For example, consider $\sigma =0.07$. Are solutions very different when
uncertainty is high? In which case does the approximation in logs make more
sense than approximation in levels (hint: have a look at the consumption
series for the models in levels)?
\end{enumerate}

\section{IRFs for non-linear models}

\noindent For linear models the IRF is unique. For non-linear models the
responses to a shock depend on (i) the values the state variables take on,
(ii) the values \emph{future} shocks take on (iii) they depend on the sign
and the size of the shock even when the responses are scaled by the shock. 

The file \texttt{IRFRBClevels2nd.m} computes the impulse responses for the
three variables in the model.

\begin{enumerate}
\item Run the file \texttt{IRFRBClevels2nd.m} with the given parameters. It
generates three impulse responses. Do not close the figure window.

\item Now change the starting values for the computation of the impulse
response. For example, multiply the starting value for capital (k0) by 1.5
(so that the computation of the impulse response will start with capital
50\% above the steady state). Compare the results with those you obtained in
part \#1. Are they the same?

\item Now undo the change made in part \#2 and run the file again (that is
do part \#1 again). Again keep the figure window open. Now change the seed
used to generate the random numbers. You'll see that there are some
differences, although small (an easy way to see them is to click on the icon
'Data Cursor' on the figure an then click on one of the lines in the
figure). Think why these differences occur. What would you have to do to if
you wanted to obtain a 'typical' impulse-response function for the second
order approximation?
\end{enumerate}

\section{A simple investment technology shock model}

\noindent Consider the following model with investment-specific technology
shocks. 
\begin{gather*}
\max_{\left\{ c_{t},i_{t},k_{t}\right\} }\hspace{2mm}\displaystyle\text{E}%
\sum_{t=1}^{\infty }\beta ^{t}\frac{c_{t}^{1-\nu }}{1-\nu }, \\
\text{s.t.}
\end{gather*}

\begin{eqnarray*}
c_t + i_t = k_{t-1}^{\alpha}
\end{eqnarray*}
\begin{eqnarray*}
k_t=(1-\delta)k_{t-1} + exp(z_t)i_t
\end{eqnarray*}
\begin{eqnarray*}
z_t=\rho z_{t-1} + e_t
\end{eqnarray*}

\noindent \textbf{First-order condition:} 
\begin{equation*}
\frac{c_{t}^{-\nu }}{exp(z_{t})}=\text{E}_{t}\left[ \beta c_{t+1}^{-\nu
}\left( \alpha k_{t}^{\alpha -1}+\frac{1-\delta }{exp(z_{t+1})}\right) %
\right] 
\end{equation*}

\noindent The purpose of this exercise is to investigate the differences
between investment technology shocks and productivity shocks, and especially
of the consequences they have for the comovement between variables. The file
to solve this model with Dynare can be found in \texttt{RBCInvestment.mod}.
The corresponding program for the model with productivity shocks is given in 
\texttt{RBCTechnology.mod}).\footnote{%
Note that the model with productivity shocks is exactly the same as the
model \texttt{RBClevels1st.mod}, with the only difference that investment
series has been made explicit.} The file \texttt{Correlations.m} \ runs both
models and computes business cycle statistics. 

You are asked to do the following:

\begin{enumerate}
\item In \texttt{Correlations.m} the part that computes artificial time
series for technology shocks has alredy been written. To investigate
investment technology shocks, write the second part that (1) runs \texttt{%
RBCInvestment.mod} using Dynare, (2) loads the coefficient matrix (decision
rules) and (3) generates artificial time series for output, consumption, and
investment. Hint: you can draw heavily on what has already been written in
the first part of \texttt{Correlations.m}.

\item When you complete part \#1, the program will display the Dynare
impulse responses and display correlations between output, consumption, and
investment for the two models. Using US data we observe that corr(c,y)=0.78
and corr(i,y)=0.88. Does the model with investment technology shocks come
close to these values? Think how you can explain the negative correlation
between consumption and investment by looking at the impulse responses of
the model with investment technology shocks.
\end{enumerate}

\end{document}
