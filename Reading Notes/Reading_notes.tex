% This is a simple template for a LaTeX document using the "article" class.
% See "book", "report", "letter" for other types of document.

\documentclass[10pt]{article} % use larger type; default would be 10pt

\usepackage[utf8]{inputenc} % set input encoding (not needed with XeLaTeX)

%%% Examples of Article customizations
% These packages are optional, depending whether you want the features they provide.
% See the LaTeX Companion or other references for full information.

%%% PAGE DIMENSIONS
\usepackage{geometry} % to change the page dimensions
\geometry{a4paper} % or letterpaper (US) or a5paper or....
\usepackage{setspace}
\usepackage{parskip}
\parskip = 0.3 \baselineskip %\advance\parskip by 0pt plus 2pt% to change between paragraphs space
% \geometry{margin=2in} % for example, change the margins to 2 inches all round
% \geometry{landscape} % set up the page for landscape
%   read geometry.pdf for detailed page layout information

% \usepackage{gravarphicx} % support the \includegravarphics command and options
% \usepackage[parfill]{parskip} % Activate to begin paragraphs with an empty line rather than an indent

%%% PACKAGES
\usepackage{booktabs} % for much better looking tables
\usepackage{array} % for better arrays (eg matrices) in maths
\usepackage{paralist} % very flexible & customisable lists (eg. enumerate/itemize, etc.)
\usepackage{verbatim} % adds environment for commenting out blocks of text & for better verbatim
\usepackage{subfig} % make it possible to include more than one captioned figure/table in a single float
% These packages are all incorporated in the memoir class to one degree or another...
\usepackage[fleqn]{amsmath}
\usepackage{amssymb}
\usepackage{enumitem}
\usepackage{amsthm}
\usepackage{graphicx}
\usepackage{filecontents}
\usepackage{natbib}
\usepackage{blindtext}
\usepackage{titlesec}
\usepackage[table,xcdraw]{xcolor}


%%% HEADERS & FOOTERS
\usepackage{fancyhdr} % This should be set AFTER setting up the page geometry
\pagestyle{plain} % options: empty , plain , fancy
\renewcommand{\headrulewidth}{0pt} % customise the layout...
\lhead{}\chead{}\rhead{}
\lfoot{}\cfoot{\thepage}\rfoot{}

%%% SECTION TITLE APPEARANCE
\usepackage{sectsty}
\allsectionsfont{\rmfamily\bfseries\upshape} % (See the fntguide.pdf for font help)
% (This matches ConTeXt defaults)

%%% ToC (table of contents) APPEARANCE
\usepackage[nottoc,notlof,notlot]{tocbibind} % Put the bibliography in the ToC
\usepackage[titles,subfigure]{tocloft} % Alter the style of the Table of Contents
\renewcommand{\cftsecfont}{\rmfamily\mdseries\upshape}
\renewcommand{\cftsecpagefont}{\rmfamily\mdseries\upshape} % No bold!

\usepackage[colorlinks,citecolor=black,urlcolor=black,bookmarks=false,hypertexnames=true]{hyperref} 
%%% END Article customizations



%%% The "real" document content comes below...

\title{Paper Reading Notes}
\author{Xing Mingjie}
\date{\today} % Activate to display a given date or no date (if empty),
         % otherwise the current date is printed 

\begin{document}
\maketitle

\tableofcontents

\newpage

\section{Trade}
    \subsection{\cite{AkcigitAtesImpullitti2018}}
    
	\subsection{\cite{Fernandesetal2023}}
    
    \subsection{\cite{MartinMejeanParenti2023}}

    \subsection{\cite{DemidovaRodriguezClare2009}}
    \begin{itemize}
        \item Welfare: productivity, terms of trade, variety, curvature (heteorogenity across varieties)
        \item consumption subsidy, export tax, import tariff allows small economy to deal with two distortions and reach first-best allocation
        \item export subsidy generates increase in productivity, but negative on other three, decrease welfare
        \item import tariff improves small economy's welfare
    \end{itemize}

    \subsection{\cite{Demidova2017}}
    \begin{itemize}
        \item Monopolistic competition with heterogenous firms, endogenous wages, non-separable, non-homothetic quadratic preferences generating variable markups
        \item optimal level of the revenue generating import tariff is strictly positive
        \item reductions in cost-shifting trade barriers are welfare-improving
        \item in both cases, variable markups result in negative pro-competitive effects, reducing gains from trade
    \end{itemize}

    \subsection{\cite{CostinotRodriguezClareIvan2020}}
    \begin{itemize}
        \item Large firms tend to export
        \item country maximizing domestic welfare, self-selection of heterogenous firms into exports calls for import subsidies on the least profitable foreign firms
        \item there is no rationale for export subsidies or taxes on the least profitable domestic firms
    \end{itemize}

    \subsection{\cite{AtkesonBurstein2008}}

    \subsection{\cite{Steinwender2018}}
    \begin{itemize}
        \item Transatlantic telegraph in 1866 lowers average and volatility of the transatlantic price difference of cotton, and increases those of trade flows. 
        \item Efficiency gains 8\% of export value
        \item A partial equilibrium model in which exporters and storage use the latest news about a foreign market to forecast expected prices.
        \item Newly collected data set on cotton prices, trade and information flows from historical newspapers.
    \end{itemize}

\section{Industrial Policy}
    \subsection{\cite{JuhaszLaneRodrik2023}}

    \subsection{\cite{MazzucatoRodrik2023}}
    \begin{itemize}
        \item Table of Taxonomy of conditionalities in the case studies
            \begin{itemize}
                \item Type of firm behavior targeted: access, directionality, profit sharing, reinvestment
                \item Fixed versus negotiable/ iterative conditions
                \item Risks/ rewards sharing mechanism
                \item Measurable performance criteria and monitoring and evaluation
            \end{itemize}
        \item Embeddedness, autonomy and the development state matrix
    \end{itemize}
    % \begin{table}[]
    %     \begin{tabular}{llllll}
    %     \hline
    %     \cellcolor[HTML]{FFFFFF}Case study &
    %       \cellcolor[HTML]{FFFFFF}Time period &
    %       \cellcolor[HTML]{FFFFFF}Policy domain &
    %       \cellcolor[HTML]{FFFFFF}Policy objectives &
    %       Nature of government incentives &
    %       Actors involved \\ \hline
    %     \cellcolor[HTML]{FFFFFF}KfW Energy Efficient Refurbishment and   Construction Programs (Germany) &
    %       \cellcolor[HTML]{FFFFFF}2009-20211 &
    %       \cellcolor[HTML]{FFFFFF}Environment, construction &
    %       \cellcolor[HTML]{FFFFFF}Support energy- efficient new constructions and improve the energy   efficiency of existing buildings &
    %       Public Bank concessional loans, progressive debt relief, grants &
    %       Government, Public Bank, private companies, homeowners, municipalities,   municipally- owned companies, independent expert verifiers \\
    %     \cellcolor[HTML]{FFFFFF}Israel High-Tech R\&D Investment   Incentives (Israel) &
    %       \cellcolor[HTML]{FFFFFF}1980-Present &
    %       \cellcolor[HTML]{FFFFFF}Technology- innovation &
    %       \cellcolor[HTML]{FFFFFF}Support for research and product development in the technology sector &
    %       R\&D grants &
    %       Government, local government, private companies, local universities \\
    %     \cellcolor[HTML]{FFFFFF}ScotWind (Scotland, UK) &
    %       \cellcolor[HTML]{FFFFFF}2021-Present &
    %       \cellcolor[HTML]{FFFFFF}Renewable energy &
    %       \cellcolor[HTML]{FFFFFF}Support the development of offshore wind industry in Scotland &
    %       Lease agreements, public bank loans &
    %       Government, local government, public banks, private companies, local   communities, state- created business development corporation \\
    %     \cellcolor[HTML]{FFFFFF}Oxford/ AstraZeneca (UK) &
    %       \cellcolor[HTML]{FFFFFF}2010-2018: R\&D technology support 2020 - 2021: pandemic response &
    %       \cellcolor[HTML]{FFFFFF}Public health (vaccine development) &
    %       \cellcolor[HTML]{FFFFFF}Create a vaccine response to COVID-19 for the UK &
    %       Grants, purchase guarantee &
    %       Government, universities, private companies \\
    %     \cellcolor[HTML]{FFFFFF}Italy's Law 488/92 Regional Investment   Subsidies (Italy) &
    %       \cellcolor[HTML]{FFFFFF}1996-2007 &
    %       \cellcolor[HTML]{FFFFFF}Manufacturing, tourism, transportation &
    %       \cellcolor[HTML]{FFFFFF}Stimulate economic growth and job creation in lagging regions &
    %       Subsidies &
    %       Government, regional government, private companies, local communities \\
    %     UK Regional Selective Assistance (UK) &
    %       1997-2020 &
    %       Manufacturing &
    %       Create and safeguard employment in areas with low economic growth &
    %       Discretionary grants &
    %       Government, regional government, private companies, local communities \\
    %     South Korean HIC Incentive (South Korea) &
    %       1970s &
    %       Structural transformation / export promotion (heavy industries) &
    %       Export promotion in six strategic sectors: steel, nonferrous metals,   shipbuilding, machinery, electronics, and petrochemicals &
    %       Subsidies, low- interest loans, export credit, tax exemption,   depreciation allowances, wastage allowances, tariff exemptions, and   concessional credits &
    %       Government, private companies, public banks, commercial banks, trade   promotion corporation \\
    %     ARPA-E (USA) &
    %       2007-Present &
    %       Technology, innovation, energy &
    %       Support lab-to-market research in new technologies for the energy sector &
    %       Grants, contracts, cash prizes and other transactions &
    %       Government, private companies, independent advisors, universities,   national laboratories \\
    %     U.S. CHIPS Act (USA) &
    %       2022-Present &
    %       Manufacturing (semi-conductor industry) &
    %       Support domestic investments on advanced manufacturing, with a focus on   semi- conductors &
    %       Grants, concessional loans, tax credits &
    %       Government, private companies, public banks, local consortia, research   institutes
    %     \end{tabular}
    %     \caption{Summary of case studies}
    %     \label{tab:my-table}
    %     \end{table}

    \subsection{\cite{Rodrik2018}}

    \subsection{\cite{AigingerRodrik2020}}
    \begin{itemize}
        \item Chronique of industrial policy definition since 1981
    \end{itemize}

    \subsection{\cite{McMillanRodrikSepulveda2017}}
    \begin{itemize}
        \item Two theories explaining growth: 1) \textbf{dual-economy.} draws distinction between agriculture as traditional and industry as modern sectors of economy. Different economic logics are at work within so cannot be lumped together. Accumulation innovation and productivity growth take place in the modern sector, the traditional sector remains technologically backward and stagnant. Labor and other resources migration rate to modern sector decides growth rate. Lewis 1954, Ranis and Fei 1961.\\
        2) \textbf{neoclassical growth model}. presumes different economic activity are structurally similar enough to aggregate into a representative sector. growth depends on the incentives to save, accumulate physical and human capital, and innovate by developing new products and processes. Solow 1956, Grossman and Helpman 1991, Aghion and Howitt 1992.
        \item two challenge: structural transformation and fundamentals. \textbf{Former}, ensure resources flow rapidly to high productivity. \textbf{Latter}, on broad and long-run growth two driving forces: quality of institutions(governance, law, biz environment) or the level of human capital(education, skills, training). Acemoglu Johnson Robinson 2001, Glaeser et al 2004.
        \item Results
        \begin{itemize}
            \item Brazil and Botswana: structural change important in launching into middle-income but tiny role thereafter
            \item Vietnam and Ghana: structural change significant contribution
            \item India, Nigeria Zambia: structural change different way. less rapid decline in the employment share of low-productivity agriculture, exacerbated by the lack of labor-intensive manufacturing for export.
        \end{itemize}
        \item Typology of growth patterns: structural transformation $\times$ Investment in fundamentals
        \item total labor productivity:\(P_t = \sum\limits_{i=1}^{n}\theta_{i,t}P_{i,t}\). \\ Change in total labor productivity \(\Delta P_t = \sum\limits_{i=1}^{n}\theta_{i,t-k}\Delta P_{i,t} + \sum\limits_{i=1}^{n}\Delta P_{i,t-k} + \sum\limits_{i=1}^{n}\Delta\theta_{i,t}\Delta P_{i,t}\)
        \item productivity change as sum of with-in sector change and structural change \[\Delta P_t = \sum\limits_{i=1}^{n}\theta_{i,t-k}\Delta P_{i,t} + \sum\limits_{i=1}^{n}P_{i,t}\Delta \theta_{i,t}\]
    \end{itemize}

\section{Innovation}
    \subsection{\cite{AghionHarrisHowittVickers2001}}
    
    \subsection{\cite{Prato2022}}
    \begin{itemize}
        \item Meeting rate
    \end{itemize}

    \subsection{\cite{BaharRapoport2018}}
    \begin{itemize}
        \item Proxy for knowledge diffusion: cross-country productivity spillovers leading to new exports
        \item 10\% increase in immigration from exporters is associated with a 2\% increase in the host country exporting in next decade likelihood, especially stronger for highly-skilled migrants.
    \end{itemize}
    \subsection{\cite{BaiJinLu2023}}

    \subsection{\cite{AdaoBerajaPandalaiNayar2020}}

    \subsection{\cite{LiuMa2023}}

    \subsection{\cite{KoganPapanikolaouSeruStoffman2017}}
    \begin{itemize}
        \item Technological innovation accounts for significant medium-run fluctuations in aggregate economic growth adn TFP.
        \item patent-level estimates of private economic value are positively related to the scientific value of these patents
        \item Extended data: \url{https://github.com/KPSS2017/Technological-Innovation-Resource-Allocation-and-Growth-Extended-Data}
    \end{itemize}

    \subsection{\cite{AghionBloomBlundellGriffithHowitt2005}}
    \begin{itemize}
        \item Competition discourages laggard firms from innovating and encourages neck-and-neck firms. This generates and inverted-U, together with competition on equilibrium industry structure
        \item average tech distance  btw leaders and followers increases with competition
        \item the inverted-U is steeper when industries are more neck-and-neck
    \end{itemize}

    \subsection{\cite{KoenigLorenzZilibotti2016}}
    \begin{itemize}
        \item Firms endogenously choose between in-house R\&D and imitation other firm's tech subject to limits of absorptive capacities to improve productivity based on profit maximization motive
        \item closer to technological frontier face fewer imitation opportunities, more in-house
        \item BGE features persistent productivity differences even when starting from identical firms
    \end{itemize}




\section{Search and Unemployment}
    \subsection{\cite{MortensenPissarides1994}}
    \subsection{\cite{Shimer2005}}

    \subsection{\cite{HornsteinKrusellViolante2011}}

    \subsection{\cite{LenoirMartinMejean2022}}
    \begin{itemize}
        \item Search friction of new customers distort the allocation activities across heterogeneous producers in a Ricardian model of trade.
        \item Markets with high estimated frictions display less dispersion in sales btw high and low productivity firms
        \item Increase in the level of search frictions pushes out the least productive exporters while increases export sales at the top of the productivity distribution
    \end{itemize}

\section{Heterogenous Agents}
    \subsection{\cite{Aiyagari1994}}

    \subsection{\cite{KrusellSmith1998}}

\section{Firm Dynamics}
    \subsection{\cite{Hopenhayn1992}}
    \subsection{\cite{Kochen2023}}

    \subsection{\cite{ArellanoBaiZhang2012}}

    \subsection{\cite{CooleyQuadrini2001}}
    \begin{itemize}
        \item Introducing financial-market frictions in a basic model of industry dynamics with persistent shocks
        \item the combination of persistent shocks and financial frictions can account for the simultaneous dependence of firm dynamics on size (once we control for age) and on age (once we control for size).
    \end{itemize}

    \subsection{\cite{Jovanovic1982}}
    \begin{itemize}
        \item This is the seminal paper to incorporate uncertainty and learning into entrepreneurship and firm dynamics.
    \end{itemize}

    \subsection{\cite{KletteKortum2004}}

    \subsection{\cite{Melitz2003}}
\section{Uncertainty}
    \subsection{\cite{ArellanoBaiKehoe2019}}
\section{Empirical}

    \subsection{\cite{DiaoElllisMcMillanRodrik2021}}
    \begin{itemize}
        \item The poor employment performance of large firms is related to use of capital-intensive techniques associated with global trends in technology.
        \item larger firms that exhibit superior productivity performance do not expand employment much
        \item small firms that absorb employment do not experience any productivity growth. 
        \item Relatively large firms in the manufacturing sectors of Tanzania and Ethiopia are significantly more capital-intensive than what would be expected on the basis of the countries$\prime$ income levels or relative factor endowments
        \item Reasons: 1, advanced economies develop labour saving technology; 2, globalization and the spread of global value chains has had a homogenizing effect on technology adoption, and the imperative of competing with production in richer countries at similar quality level makes it difficult to undertake large shifts in techniques
        \item New panels of manufacturing firms: Tanzania 2008$\sim$ 16, Ethiopia 1996$\sim$ 2017
    \end{itemize}

    \subsection{\cite{LaplaneMazzucato2020}}
    \begin{itemize}
        \item Policies that explicitly take into consideration the risk-taking entrepreneurial role of the state, can positively affect reward distributions and favor more equitable public$\textemdash$ private partnerships.
        \item Sharing rewards enables a more portfolio mindset, where the upside is used to cover the downside, and more stable funding to serve citizens' needs. It also signals the value and legitimacy of the state’s role.
        \item Table of existing policy instruments for financing innovation that allow for profit-Sharing
        \item Table of the legal underpinning of the distribution of rewards in public\(\textemdash\)private partnerships parasitic versus symbiotic ecosystems
    \end{itemize}

    \subsection{\cite{Rodrik2016}}
    \begin{itemize}
        \item Developing countries only converge to rich country income levels conditional on country-specific disadvantages like institutions or poor geography being overcome.
        \item Matrix of structural change and investment in fundamentals
        \item much of recent performance in Africa due to advantageous external context and making up of lost ground
        \item structural change and industrialization operating at less than full power
        \item should there be a miracle, it should be agriculture or service led than traditional ones.
    \end{itemize}

    \subsection{\cite{AghionAkcigitHyytinenToivanen2018}}
    \begin{itemize}
        \item inventor collect 8\% of total private return
        \item entrepreneurs get over 44\%
        \item blue-collar get 26\%
        \item the rest goes to white-collar workers
        \item entrepreneurs have negative returns prior to patent application but subsequently become highly positive
        \item Finland data
    \end{itemize}

    \subsection{\cite{Akcigitetal2018}}
    \begin{itemize}
        \item Higher taxes negatively impact the quantity and the location of innovation, but not average innovation quality.
        \item state-level elasticities to taxes are large and consistent with the aggregation of the individual level responses of innovation produced and cross-state mobility
        \item corporate taxes have special effect on corporate inventor's innovation production and mobility
        \item personal income tax affects quantity of innovation and mobility of inventors.
        \item panel of patent inventors since 1920
        \item historical state-level corporate tax database with corp tax rates and tax base information
        \item existing: state-level personal income taxes
    \end{itemize}

\section{Migration}
    \subsection{\cite{DocquierMachadoSekkat2015}}
    \begin{itemize}
        \item Complete liberalization of cross-border migration increase world GDP by 11.5-12.5\% in benchmark model, and 7.0-17.9\% in robustness analyses.
    \end{itemize}

    \subsection{\cite{ParsonsVezina2018}}
    \begin{itemize}
        \item The exodus of Vietnamese Boat People to US evidence that migrant networks promote trade by reducing trade costs because they have knowledge of their home country's language, regulations, market opportunities and informal institutions. Migrants mostly facilitate bilateral trade with developing countries.
        \item Doubling migrants leads to 45\% to 138\% increase in state exports.
        \item First evidence of positive link between migration and trade with a natural experiment.
    \end{itemize}

    \subsection{\cite{BurchardiChaneyHassan2019}}
    \begin{itemize}
        \item Doubling the number of residents with ancestry from a given foreign country relative to the mean increases the probability that at least one local firm engages in FDI with that country by 4 percentage points.
        \item This effect is primarily driven by a reduction in information frictions, and not by better contract enforcement, taste similarities, or a convergence in factor endowments.
    \end{itemize}

    \subsection{\cite{CohenGurunMalloy2017}}
    \begin{itemize}
        \item Firms are significantly more likely to trade with countries that have a large resident population near their firm headquarters, and that these connected trades are their most valuable international trades
        \item Firms are also more likely to acquire target firms, and report increased segment sales, in connected countries
    \end{itemize}
\newpage
\section{Data}
    \subsection{Macro}
        \begin{itemize}
            \item Groningen Growth and Development Centre: indicators of growth and development \url{https://www.rug.nl/ggdc/}
        \end{itemize}

    \subsection{Medical}
        \begin{itemize}
            \item Medical Expenditure Panel Survey \url{https://meps.ahrq.gov/mepsweb/}
            \item FDA Orange book \url{https://www.fda.gov/drugs/drug-approvals-and-databases/approved-drug-products-therapeutic-equivalence-evaluations-orange-book}
        \end{itemize}

    \subsection{Trade}
        \begin{itemize}
            \item GlobalTradeAlert \url{https://www.globaltradealert.org/}
            \item Cboe Trade Alerts \url{https://www.cboe.com/services/analytics/tradealert/institutions/}
            \item UN Comtrade: product level bilateral trade data \url{https://comtradeplus.un.org/TradeFlow}
        \end{itemize}

    \subsection{Legal}
        \begin{itemize}
        \item LobbyView \url{https://www.lobbyview.org/}
        \item QuantGov \url{https://www.quantgov.org/}
        \end{itemize}
    
    \subsection{Innovation}
        \begin{itemize}
            \item Orgiin IP Solutions \url{https://origiin.com/}
            \item PATENTS-ICRIOS DATABASE \url{https://icrios.unibocconi.eu/resources/databases/patents-icrios-database}
            \item 
        \end{itemize}
    
    \subsection{Firm}
        \begin{itemize}
            \item FactSet Supply Chain Relationships \url{https://wrds-www.wharton.upenn.edu/pages/about/data-vendors/factset/}
            \item Cortellis \url{https://access.clarivate.com/login?app=cortellis}
            \item Westlaw Edge \url{https://legal.thomsonreuters.com/en/products/westlaw-edge}
            \item MIDAS \url{https://www.iqvia.com/solutions/commercialization/brand-strategy-and-management/market-measurement/midas}
            \item Namsor \url{https://namsor.app/}
            \item GGDC 10 Sector database
            \item expanded africa sector database
            \item UNIDO's Indstat2
        \end{itemize}
    
    \subsection{Transport}
        \begin{itemize}
            \item TUD19: multi-city traffic dataset \url{https://utd19.ethz.ch/}; \url{https://github.com/ambuehll/UTD19}
        \end{itemize}

    \subsection{Migration}
        \begin{itemize}
            \item Gallup Global Research\url{https://www.gallup.com/analytics/318875/global-research.aspx}
            \item Gallup Country Data Set \url{https://www.gallup.com/services/177797/country-data-set-details.aspx}
            \item Bilateral migration data \cite{Artucetal2014}
        \end{itemize}

    \subsection{Others}
        \begin{itemize}
            \item GeoDist on bilateral relationships such as common coloniser, colony-coloniser, common language, religion. \url{http://www.cepii.fr/CEPII/en/bdd_modele/bdd_modele_item.asp?id=6}
        \end{itemize}

\newpage

\footnotesize
\bibliographystyle{apalike}
\bibliography{Reading_notes}

\end{document}